\begin{resumo}



A tecnologia de RFID (\textit{Radio-Frequency Identification}) tem sido utilizada em várias aplicações, tais como controle de acesso, pagamento de transporte, rastreamento de produtos, animais e veículos, pelo fato de oferecer uma alternativa de baixo custo para identificação de pessoas e objetos, sem necessitar de um operador para capturar os dados. Um sistema típico de RFID possui um leitor (\textit{reader}) que envia um sinal para uma etiqueta (\textit{tag}), recebe a resposta correspondente e armazena as informações em uma base de dados. Uma \textit{tag} típica, por sua vez, contém antena, modulador, demodulador e memória, que pode, por exemplo, armazenar a informação de identificação do objeto a ser enviada para o leitor.

Trabalhos anteriores de graduação abordaram o projeto e implementação de uma \textit{tag} passiva de RFID de 13,56 MHz, sendo que o bloco de \textit{front-end} analógico/RF foi modelado em Verilog-AMS e implementado em  tecnologia TSMC 0.18 um. Os blocos digitais foram modelados em Verilog e simulados isoladamente, considerando que a comunicação entre o leitor e a \textit{tag} utiliza o protocolo ISO/IEC 14443 e que inicialmente a \textit{tag} envia somente o número de identificação para o leitor. Considerando esse contexto, este projeto tem como objetivo geral a verificação dos blocos digitais da \textit{tag} em conjunto com o sistema projetado nos trabalhos anteriores através de simulações mistas e análises comparativas de desempenho.

 \vspace{\onelineskip}
    
 \noindent
 \textbf{Palavras-chaves}: RFID, Circuitos Integrados, Verificação, Simulação Mista.
\end{resumo}
