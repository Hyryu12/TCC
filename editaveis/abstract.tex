\begin{resumo}[Abstract]
 \begin{otherlanguage*}{english}
   
   The RFID technology has been used in different applications, such as access control, transportation payment, tracking of products, animals and vehicles because it can offer a low-cost option in the field of identification of persons and objects without needing an operator to capture data. A typical RFID system is composed of a reader and a tag. The reader sends a signal to the tag, receives the corresponding answer and stores the information in a data bank. The tag is equipped with a memory chip, which stores the identification of the object. This information is read and sent back to the reader. A typical tag has an antenna, modulator, demodulator and memory chip.
   
      Previous graduation work approached the project and implementation of a 13,56 MHz passive tag, with the analogic/RF front-end block being modelled in Verilog-AMS and implemented on  0.18 um TSMC technology. The digital blocks were modelled in Verilog and simulated, considering that the communication between the reader and the tag uses the ISO/IEC 14443 protocol and that initially the tag sends only the identification number to the reader.
      
      In this context, the aim of this project is the design of digital blocks that will be used for the verification of the system designed in previous work through mixed-signal simulations and comparative performance analysis
   

   \vspace{\onelineskip}
 
   \noindent 
   \textbf{Key-words}: RFID, Integrated Circuits, Verification, Mixed-Signal Simulation.
 \end{otherlanguage*}
\end{resumo}
